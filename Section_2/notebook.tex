
% Default to the notebook output style

    


% Inherit from the specified cell style.




    
\documentclass[11pt]{article}

    
    
    \usepackage[T1]{fontenc}
    % Nicer default font (+ math font) than Computer Modern for most use cases
    \usepackage{mathpazo}

    % Basic figure setup, for now with no caption control since it's done
    % automatically by Pandoc (which extracts ![](path) syntax from Markdown).
    \usepackage{graphicx}
    % We will generate all images so they have a width \maxwidth. This means
    % that they will get their normal width if they fit onto the page, but
    % are scaled down if they would overflow the margins.
    \makeatletter
    \def\maxwidth{\ifdim\Gin@nat@width>\linewidth\linewidth
    \else\Gin@nat@width\fi}
    \makeatother
    \let\Oldincludegraphics\includegraphics
    % Set max figure width to be 80% of text width, for now hardcoded.
    \renewcommand{\includegraphics}[1]{\Oldincludegraphics[width=.8\maxwidth]{#1}}
    % Ensure that by default, figures have no caption (until we provide a
    % proper Figure object with a Caption API and a way to capture that
    % in the conversion process - todo).
    \usepackage{caption}
    \DeclareCaptionLabelFormat{nolabel}{}
    \captionsetup{labelformat=nolabel}

    \usepackage{adjustbox} % Used to constrain images to a maximum size 
    \usepackage{xcolor} % Allow colors to be defined
    \usepackage{enumerate} % Needed for markdown enumerations to work
    \usepackage{geometry} % Used to adjust the document margins
    \usepackage{amsmath} % Equations
    \usepackage{amssymb} % Equations
    \usepackage{textcomp} % defines textquotesingle
    % Hack from http://tex.stackexchange.com/a/47451/13684:
    \AtBeginDocument{%
        \def\PYZsq{\textquotesingle}% Upright quotes in Pygmentized code
    }
    \usepackage{upquote} % Upright quotes for verbatim code
    \usepackage{eurosym} % defines \euro
    \usepackage[mathletters]{ucs} % Extended unicode (utf-8) support
    \usepackage[utf8x]{inputenc} % Allow utf-8 characters in the tex document
    \usepackage{fancyvrb} % verbatim replacement that allows latex
    \usepackage{grffile} % extends the file name processing of package graphics 
                         % to support a larger range 
    % The hyperref package gives us a pdf with properly built
    % internal navigation ('pdf bookmarks' for the table of contents,
    % internal cross-reference links, web links for URLs, etc.)
    \usepackage{hyperref}
    \usepackage{longtable} % longtable support required by pandoc >1.10
    \usepackage{booktabs}  % table support for pandoc > 1.12.2
    \usepackage[inline]{enumitem} % IRkernel/repr support (it uses the enumerate* environment)
    \usepackage[normalem]{ulem} % ulem is needed to support strikethroughs (\sout)
                                % normalem makes italics be italics, not underlines
    

    
    
    % Colors for the hyperref package
    \definecolor{urlcolor}{rgb}{0,.145,.698}
    \definecolor{linkcolor}{rgb}{.71,0.21,0.01}
    \definecolor{citecolor}{rgb}{.12,.54,.11}

    % ANSI colors
    \definecolor{ansi-black}{HTML}{3E424D}
    \definecolor{ansi-black-intense}{HTML}{282C36}
    \definecolor{ansi-red}{HTML}{E75C58}
    \definecolor{ansi-red-intense}{HTML}{B22B31}
    \definecolor{ansi-green}{HTML}{00A250}
    \definecolor{ansi-green-intense}{HTML}{007427}
    \definecolor{ansi-yellow}{HTML}{DDB62B}
    \definecolor{ansi-yellow-intense}{HTML}{B27D12}
    \definecolor{ansi-blue}{HTML}{208FFB}
    \definecolor{ansi-blue-intense}{HTML}{0065CA}
    \definecolor{ansi-magenta}{HTML}{D160C4}
    \definecolor{ansi-magenta-intense}{HTML}{A03196}
    \definecolor{ansi-cyan}{HTML}{60C6C8}
    \definecolor{ansi-cyan-intense}{HTML}{258F8F}
    \definecolor{ansi-white}{HTML}{C5C1B4}
    \definecolor{ansi-white-intense}{HTML}{A1A6B2}

    % commands and environments needed by pandoc snippets
    % extracted from the output of `pandoc -s`
    \providecommand{\tightlist}{%
      \setlength{\itemsep}{0pt}\setlength{\parskip}{0pt}}
    \DefineVerbatimEnvironment{Highlighting}{Verbatim}{commandchars=\\\{\}}
    % Add ',fontsize=\small' for more characters per line
    \newenvironment{Shaded}{}{}
    \newcommand{\KeywordTok}[1]{\textcolor[rgb]{0.00,0.44,0.13}{\textbf{{#1}}}}
    \newcommand{\DataTypeTok}[1]{\textcolor[rgb]{0.56,0.13,0.00}{{#1}}}
    \newcommand{\DecValTok}[1]{\textcolor[rgb]{0.25,0.63,0.44}{{#1}}}
    \newcommand{\BaseNTok}[1]{\textcolor[rgb]{0.25,0.63,0.44}{{#1}}}
    \newcommand{\FloatTok}[1]{\textcolor[rgb]{0.25,0.63,0.44}{{#1}}}
    \newcommand{\CharTok}[1]{\textcolor[rgb]{0.25,0.44,0.63}{{#1}}}
    \newcommand{\StringTok}[1]{\textcolor[rgb]{0.25,0.44,0.63}{{#1}}}
    \newcommand{\CommentTok}[1]{\textcolor[rgb]{0.38,0.63,0.69}{\textit{{#1}}}}
    \newcommand{\OtherTok}[1]{\textcolor[rgb]{0.00,0.44,0.13}{{#1}}}
    \newcommand{\AlertTok}[1]{\textcolor[rgb]{1.00,0.00,0.00}{\textbf{{#1}}}}
    \newcommand{\FunctionTok}[1]{\textcolor[rgb]{0.02,0.16,0.49}{{#1}}}
    \newcommand{\RegionMarkerTok}[1]{{#1}}
    \newcommand{\ErrorTok}[1]{\textcolor[rgb]{1.00,0.00,0.00}{\textbf{{#1}}}}
    \newcommand{\NormalTok}[1]{{#1}}
    
    % Additional commands for more recent versions of Pandoc
    \newcommand{\ConstantTok}[1]{\textcolor[rgb]{0.53,0.00,0.00}{{#1}}}
    \newcommand{\SpecialCharTok}[1]{\textcolor[rgb]{0.25,0.44,0.63}{{#1}}}
    \newcommand{\VerbatimStringTok}[1]{\textcolor[rgb]{0.25,0.44,0.63}{{#1}}}
    \newcommand{\SpecialStringTok}[1]{\textcolor[rgb]{0.73,0.40,0.53}{{#1}}}
    \newcommand{\ImportTok}[1]{{#1}}
    \newcommand{\DocumentationTok}[1]{\textcolor[rgb]{0.73,0.13,0.13}{\textit{{#1}}}}
    \newcommand{\AnnotationTok}[1]{\textcolor[rgb]{0.38,0.63,0.69}{\textbf{\textit{{#1}}}}}
    \newcommand{\CommentVarTok}[1]{\textcolor[rgb]{0.38,0.63,0.69}{\textbf{\textit{{#1}}}}}
    \newcommand{\VariableTok}[1]{\textcolor[rgb]{0.10,0.09,0.49}{{#1}}}
    \newcommand{\ControlFlowTok}[1]{\textcolor[rgb]{0.00,0.44,0.13}{\textbf{{#1}}}}
    \newcommand{\OperatorTok}[1]{\textcolor[rgb]{0.40,0.40,0.40}{{#1}}}
    \newcommand{\BuiltInTok}[1]{{#1}}
    \newcommand{\ExtensionTok}[1]{{#1}}
    \newcommand{\PreprocessorTok}[1]{\textcolor[rgb]{0.74,0.48,0.00}{{#1}}}
    \newcommand{\AttributeTok}[1]{\textcolor[rgb]{0.49,0.56,0.16}{{#1}}}
    \newcommand{\InformationTok}[1]{\textcolor[rgb]{0.38,0.63,0.69}{\textbf{\textit{{#1}}}}}
    \newcommand{\WarningTok}[1]{\textcolor[rgb]{0.38,0.63,0.69}{\textbf{\textit{{#1}}}}}
    
    
    % Define a nice break command that doesn't care if a line doesn't already
    % exist.
    \def\br{\hspace*{\fill} \\* }
    % Math Jax compatability definitions
    \def\gt{>}
    \def\lt{<}
    % Document parameters
    \title{Section\_2}
    
    
    

    % Pygments definitions
    
\makeatletter
\def\PY@reset{\let\PY@it=\relax \let\PY@bf=\relax%
    \let\PY@ul=\relax \let\PY@tc=\relax%
    \let\PY@bc=\relax \let\PY@ff=\relax}
\def\PY@tok#1{\csname PY@tok@#1\endcsname}
\def\PY@toks#1+{\ifx\relax#1\empty\else%
    \PY@tok{#1}\expandafter\PY@toks\fi}
\def\PY@do#1{\PY@bc{\PY@tc{\PY@ul{%
    \PY@it{\PY@bf{\PY@ff{#1}}}}}}}
\def\PY#1#2{\PY@reset\PY@toks#1+\relax+\PY@do{#2}}

\expandafter\def\csname PY@tok@w\endcsname{\def\PY@tc##1{\textcolor[rgb]{0.73,0.73,0.73}{##1}}}
\expandafter\def\csname PY@tok@c\endcsname{\let\PY@it=\textit\def\PY@tc##1{\textcolor[rgb]{0.25,0.50,0.50}{##1}}}
\expandafter\def\csname PY@tok@cp\endcsname{\def\PY@tc##1{\textcolor[rgb]{0.74,0.48,0.00}{##1}}}
\expandafter\def\csname PY@tok@k\endcsname{\let\PY@bf=\textbf\def\PY@tc##1{\textcolor[rgb]{0.00,0.50,0.00}{##1}}}
\expandafter\def\csname PY@tok@kp\endcsname{\def\PY@tc##1{\textcolor[rgb]{0.00,0.50,0.00}{##1}}}
\expandafter\def\csname PY@tok@kt\endcsname{\def\PY@tc##1{\textcolor[rgb]{0.69,0.00,0.25}{##1}}}
\expandafter\def\csname PY@tok@o\endcsname{\def\PY@tc##1{\textcolor[rgb]{0.40,0.40,0.40}{##1}}}
\expandafter\def\csname PY@tok@ow\endcsname{\let\PY@bf=\textbf\def\PY@tc##1{\textcolor[rgb]{0.67,0.13,1.00}{##1}}}
\expandafter\def\csname PY@tok@nb\endcsname{\def\PY@tc##1{\textcolor[rgb]{0.00,0.50,0.00}{##1}}}
\expandafter\def\csname PY@tok@nf\endcsname{\def\PY@tc##1{\textcolor[rgb]{0.00,0.00,1.00}{##1}}}
\expandafter\def\csname PY@tok@nc\endcsname{\let\PY@bf=\textbf\def\PY@tc##1{\textcolor[rgb]{0.00,0.00,1.00}{##1}}}
\expandafter\def\csname PY@tok@nn\endcsname{\let\PY@bf=\textbf\def\PY@tc##1{\textcolor[rgb]{0.00,0.00,1.00}{##1}}}
\expandafter\def\csname PY@tok@ne\endcsname{\let\PY@bf=\textbf\def\PY@tc##1{\textcolor[rgb]{0.82,0.25,0.23}{##1}}}
\expandafter\def\csname PY@tok@nv\endcsname{\def\PY@tc##1{\textcolor[rgb]{0.10,0.09,0.49}{##1}}}
\expandafter\def\csname PY@tok@no\endcsname{\def\PY@tc##1{\textcolor[rgb]{0.53,0.00,0.00}{##1}}}
\expandafter\def\csname PY@tok@nl\endcsname{\def\PY@tc##1{\textcolor[rgb]{0.63,0.63,0.00}{##1}}}
\expandafter\def\csname PY@tok@ni\endcsname{\let\PY@bf=\textbf\def\PY@tc##1{\textcolor[rgb]{0.60,0.60,0.60}{##1}}}
\expandafter\def\csname PY@tok@na\endcsname{\def\PY@tc##1{\textcolor[rgb]{0.49,0.56,0.16}{##1}}}
\expandafter\def\csname PY@tok@nt\endcsname{\let\PY@bf=\textbf\def\PY@tc##1{\textcolor[rgb]{0.00,0.50,0.00}{##1}}}
\expandafter\def\csname PY@tok@nd\endcsname{\def\PY@tc##1{\textcolor[rgb]{0.67,0.13,1.00}{##1}}}
\expandafter\def\csname PY@tok@s\endcsname{\def\PY@tc##1{\textcolor[rgb]{0.73,0.13,0.13}{##1}}}
\expandafter\def\csname PY@tok@sd\endcsname{\let\PY@it=\textit\def\PY@tc##1{\textcolor[rgb]{0.73,0.13,0.13}{##1}}}
\expandafter\def\csname PY@tok@si\endcsname{\let\PY@bf=\textbf\def\PY@tc##1{\textcolor[rgb]{0.73,0.40,0.53}{##1}}}
\expandafter\def\csname PY@tok@se\endcsname{\let\PY@bf=\textbf\def\PY@tc##1{\textcolor[rgb]{0.73,0.40,0.13}{##1}}}
\expandafter\def\csname PY@tok@sr\endcsname{\def\PY@tc##1{\textcolor[rgb]{0.73,0.40,0.53}{##1}}}
\expandafter\def\csname PY@tok@ss\endcsname{\def\PY@tc##1{\textcolor[rgb]{0.10,0.09,0.49}{##1}}}
\expandafter\def\csname PY@tok@sx\endcsname{\def\PY@tc##1{\textcolor[rgb]{0.00,0.50,0.00}{##1}}}
\expandafter\def\csname PY@tok@m\endcsname{\def\PY@tc##1{\textcolor[rgb]{0.40,0.40,0.40}{##1}}}
\expandafter\def\csname PY@tok@gh\endcsname{\let\PY@bf=\textbf\def\PY@tc##1{\textcolor[rgb]{0.00,0.00,0.50}{##1}}}
\expandafter\def\csname PY@tok@gu\endcsname{\let\PY@bf=\textbf\def\PY@tc##1{\textcolor[rgb]{0.50,0.00,0.50}{##1}}}
\expandafter\def\csname PY@tok@gd\endcsname{\def\PY@tc##1{\textcolor[rgb]{0.63,0.00,0.00}{##1}}}
\expandafter\def\csname PY@tok@gi\endcsname{\def\PY@tc##1{\textcolor[rgb]{0.00,0.63,0.00}{##1}}}
\expandafter\def\csname PY@tok@gr\endcsname{\def\PY@tc##1{\textcolor[rgb]{1.00,0.00,0.00}{##1}}}
\expandafter\def\csname PY@tok@ge\endcsname{\let\PY@it=\textit}
\expandafter\def\csname PY@tok@gs\endcsname{\let\PY@bf=\textbf}
\expandafter\def\csname PY@tok@gp\endcsname{\let\PY@bf=\textbf\def\PY@tc##1{\textcolor[rgb]{0.00,0.00,0.50}{##1}}}
\expandafter\def\csname PY@tok@go\endcsname{\def\PY@tc##1{\textcolor[rgb]{0.53,0.53,0.53}{##1}}}
\expandafter\def\csname PY@tok@gt\endcsname{\def\PY@tc##1{\textcolor[rgb]{0.00,0.27,0.87}{##1}}}
\expandafter\def\csname PY@tok@err\endcsname{\def\PY@bc##1{\setlength{\fboxsep}{0pt}\fcolorbox[rgb]{1.00,0.00,0.00}{1,1,1}{\strut ##1}}}
\expandafter\def\csname PY@tok@kc\endcsname{\let\PY@bf=\textbf\def\PY@tc##1{\textcolor[rgb]{0.00,0.50,0.00}{##1}}}
\expandafter\def\csname PY@tok@kd\endcsname{\let\PY@bf=\textbf\def\PY@tc##1{\textcolor[rgb]{0.00,0.50,0.00}{##1}}}
\expandafter\def\csname PY@tok@kn\endcsname{\let\PY@bf=\textbf\def\PY@tc##1{\textcolor[rgb]{0.00,0.50,0.00}{##1}}}
\expandafter\def\csname PY@tok@kr\endcsname{\let\PY@bf=\textbf\def\PY@tc##1{\textcolor[rgb]{0.00,0.50,0.00}{##1}}}
\expandafter\def\csname PY@tok@bp\endcsname{\def\PY@tc##1{\textcolor[rgb]{0.00,0.50,0.00}{##1}}}
\expandafter\def\csname PY@tok@fm\endcsname{\def\PY@tc##1{\textcolor[rgb]{0.00,0.00,1.00}{##1}}}
\expandafter\def\csname PY@tok@vc\endcsname{\def\PY@tc##1{\textcolor[rgb]{0.10,0.09,0.49}{##1}}}
\expandafter\def\csname PY@tok@vg\endcsname{\def\PY@tc##1{\textcolor[rgb]{0.10,0.09,0.49}{##1}}}
\expandafter\def\csname PY@tok@vi\endcsname{\def\PY@tc##1{\textcolor[rgb]{0.10,0.09,0.49}{##1}}}
\expandafter\def\csname PY@tok@vm\endcsname{\def\PY@tc##1{\textcolor[rgb]{0.10,0.09,0.49}{##1}}}
\expandafter\def\csname PY@tok@sa\endcsname{\def\PY@tc##1{\textcolor[rgb]{0.73,0.13,0.13}{##1}}}
\expandafter\def\csname PY@tok@sb\endcsname{\def\PY@tc##1{\textcolor[rgb]{0.73,0.13,0.13}{##1}}}
\expandafter\def\csname PY@tok@sc\endcsname{\def\PY@tc##1{\textcolor[rgb]{0.73,0.13,0.13}{##1}}}
\expandafter\def\csname PY@tok@dl\endcsname{\def\PY@tc##1{\textcolor[rgb]{0.73,0.13,0.13}{##1}}}
\expandafter\def\csname PY@tok@s2\endcsname{\def\PY@tc##1{\textcolor[rgb]{0.73,0.13,0.13}{##1}}}
\expandafter\def\csname PY@tok@sh\endcsname{\def\PY@tc##1{\textcolor[rgb]{0.73,0.13,0.13}{##1}}}
\expandafter\def\csname PY@tok@s1\endcsname{\def\PY@tc##1{\textcolor[rgb]{0.73,0.13,0.13}{##1}}}
\expandafter\def\csname PY@tok@mb\endcsname{\def\PY@tc##1{\textcolor[rgb]{0.40,0.40,0.40}{##1}}}
\expandafter\def\csname PY@tok@mf\endcsname{\def\PY@tc##1{\textcolor[rgb]{0.40,0.40,0.40}{##1}}}
\expandafter\def\csname PY@tok@mh\endcsname{\def\PY@tc##1{\textcolor[rgb]{0.40,0.40,0.40}{##1}}}
\expandafter\def\csname PY@tok@mi\endcsname{\def\PY@tc##1{\textcolor[rgb]{0.40,0.40,0.40}{##1}}}
\expandafter\def\csname PY@tok@il\endcsname{\def\PY@tc##1{\textcolor[rgb]{0.40,0.40,0.40}{##1}}}
\expandafter\def\csname PY@tok@mo\endcsname{\def\PY@tc##1{\textcolor[rgb]{0.40,0.40,0.40}{##1}}}
\expandafter\def\csname PY@tok@ch\endcsname{\let\PY@it=\textit\def\PY@tc##1{\textcolor[rgb]{0.25,0.50,0.50}{##1}}}
\expandafter\def\csname PY@tok@cm\endcsname{\let\PY@it=\textit\def\PY@tc##1{\textcolor[rgb]{0.25,0.50,0.50}{##1}}}
\expandafter\def\csname PY@tok@cpf\endcsname{\let\PY@it=\textit\def\PY@tc##1{\textcolor[rgb]{0.25,0.50,0.50}{##1}}}
\expandafter\def\csname PY@tok@c1\endcsname{\let\PY@it=\textit\def\PY@tc##1{\textcolor[rgb]{0.25,0.50,0.50}{##1}}}
\expandafter\def\csname PY@tok@cs\endcsname{\let\PY@it=\textit\def\PY@tc##1{\textcolor[rgb]{0.25,0.50,0.50}{##1}}}

\def\PYZbs{\char`\\}
\def\PYZus{\char`\_}
\def\PYZob{\char`\{}
\def\PYZcb{\char`\}}
\def\PYZca{\char`\^}
\def\PYZam{\char`\&}
\def\PYZlt{\char`\<}
\def\PYZgt{\char`\>}
\def\PYZsh{\char`\#}
\def\PYZpc{\char`\%}
\def\PYZdl{\char`\$}
\def\PYZhy{\char`\-}
\def\PYZsq{\char`\'}
\def\PYZdq{\char`\"}
\def\PYZti{\char`\~}
% for compatibility with earlier versions
\def\PYZat{@}
\def\PYZlb{[}
\def\PYZrb{]}
\makeatother


    % Exact colors from NB
    \definecolor{incolor}{rgb}{0.0, 0.0, 0.5}
    \definecolor{outcolor}{rgb}{0.545, 0.0, 0.0}



    
    % Prevent overflowing lines due to hard-to-break entities
    \sloppy 
    % Setup hyperref package
    \hypersetup{
      breaklinks=true,  % so long urls are correctly broken across lines
      colorlinks=true,
      urlcolor=urlcolor,
      linkcolor=linkcolor,
      citecolor=citecolor,
      }
    % Slightly bigger margins than the latex defaults
    
    \geometry{verbose,tmargin=1in,bmargin=1in,lmargin=1in,rmargin=1in}
    
    

    \begin{document}
    
    
    \maketitle
    
    

    
    \subsection{Section 2: Python Basics}\label{section-2-python-basics}

Before you get started on anything else, you'll need to understand
Python, the programming language used throughout this tutorial. Let's go
through some of the basics of Python together.

\subsubsection{Strings \& Print()}\label{strings-print}

A \textbf{String} is a simple data type that can include letters,
numbers, and symbols. Simply speaking, it's a sequence of characters.
Strings must be surrounded by a pair of quotation marks. Here's a couple
examples:

\begin{itemize}
\tightlist
\item
  "This is a string!"
\item
  "1234 \& 5678"
\item
  "P Sherman 42 Wallaby Way Sydney"
\end{itemize}

We use the \textbf{print()} function to display outputs in the console.
In this case, the output will show up right below the coding cell.

Now, it's your turn. Using the print() function provided below, enter a
String inside the parenthesis that says \emph{Hello World!}

Click "Run" to display the output.

    \begin{Verbatim}[commandchars=\\\{\}]
{\color{incolor}In [{\color{incolor}1}]:} \PY{n+nb}{print}\PY{p}{(}\PY{l+s+s2}{\PYZdq{}}\PY{l+s+s2}{jhhuh}\PY{l+s+s2}{\PYZdq{}}\PY{p}{)} 
\end{Verbatim}


    \begin{Verbatim}[commandchars=\\\{\}]
jhhuh

    \end{Verbatim}

    \subsubsection{Comments}\label{comments}

Before we continue further, I want to show you how to comment the code
you're working with. Commenting your code makes it easier to understand,
both for yourself in the future and for other people. Comments are parts
of the code that Python will not try to run. There's two different types
of comments: Single-line, and multi-line.

\textbf{Single-Line} comments are comments that only take up one line in
your code. These kinds of comments are good if you need to include very
brief information, such as quick notes for what a part of your code
does. Single line comments are denoted with a a hashtag ( \# ).

\textbf{Multi-line} comments, on the other hand, can be more thorough
and are used for comments that take up multiple lines of code. Muti-line
comments can also be used to "comment-out" parts of your code, if you do
not want it to run. These kinds of comments are denoted by three pairs
of quotation marks ( """ ) on either side of the comment.

Here's a couple examples of each:

\begin{itemize}
\item ~
  \section{This is a single line
  comment}\label{this-is-a-single-line-comment}
\item
  """This is a multi line comment. Anything I say and/or any code I
  write will not be run by Python End comment."""
\end{itemize}

Now you try. First, run the code below to see the output. Then, add a
single-line above the first print() function in the cell below. Then,
comment-out the second and third print() funtions using a multi-line
comment. Now run it again.

    \begin{Verbatim}[commandchars=\\\{\}]
{\color{incolor}In [{\color{incolor}32}]:} \PY{n+nb}{print}\PY{p}{(}\PY{l+s+s2}{\PYZdq{}}\PY{l+s+s2}{Add your single line comment above this function}\PY{l+s+s2}{\PYZdq{}}\PY{p}{)}
         
         \PY{n+nb}{print}\PY{p}{(}\PY{l+s+s2}{\PYZdq{}}\PY{l+s+s2}{Comment this function out.}\PY{l+s+s2}{\PYZdq{}}\PY{p}{)}
         \PY{n+nb}{print}\PY{p}{(}\PY{l+s+s2}{\PYZdq{}}\PY{l+s+s2}{And this one too!}\PY{l+s+s2}{\PYZdq{}}\PY{p}{)}
         
         
         \PY{c+c1}{\PYZsh{}You see how the last two Strings do not show up when you run the code the second time? }
\end{Verbatim}


    \begin{Verbatim}[commandchars=\\\{\}]
Add your single line comment above this function
Comment this function out.
And this one too!

    \end{Verbatim}

    \subsubsection{Variables and Assignment}\label{variables-and-assignment}

One way to store and work with different types of data is by using
\textbf{variables}. Variables stores data under the name you decide to
give it. You can \textbf{assign} the variable different pieces of data
by using the equals ( = ) symbol. For example, if I wrote

myFirstVariable = 24 mySecondVariable = "Hello!"

(\textbf{\emph{Please do note that the "=" sign is for assignment, not
for comparison. The symbol for comparison is a double-equals sign ( ==
).}})

In the above examples, I assigned myFirstVariable to hold the number 24.
I assigned mySecondVariable to hold the String "Hello!" In each case,
the name (myFirstVariable/mySecondVariable) is the name of the variable,
but it holds different types of data. One holds a number, and one holds
a String.

Create two different variables in the cell below and assign them any
number or String value you want. Then, print both variables out using
two print() functions. Be sure to put the name of your variables inside
the print function in order to see it printed out on the screen.

    \begin{Verbatim}[commandchars=\\\{\}]
{\color{incolor}In [{\color{incolor}33}]:} \PY{c+c1}{\PYZsh{} This is example code below. }
         
         \PY{n}{var1} \PY{o}{=} \PY{l+s+s2}{\PYZdq{}}\PY{l+s+s2}{last}\PY{l+s+s2}{\PYZdq{}}
         \PY{n}{var2} \PY{o}{=} \PY{l+m+mi}{64}
         
         \PY{n+nb}{print}\PY{p}{(}\PY{n}{var1}\PY{p}{)}
         \PY{n+nb}{print}\PY{p}{(}\PY{n}{var2}\PY{p}{)}
         
         \PY{c+c1}{\PYZsh{}Notice that only the data each variable holds is printed out! Not the variable name itself!}
\end{Verbatim}


    \begin{Verbatim}[commandchars=\\\{\}]
last
64

    \end{Verbatim}

    \subsubsection{Operations}\label{operations}

Python allows you to perform different mathematical operations in your
code, including but not limited to: * Addition (using the + sign) *
Subtraction ( - ) * Multiplication ( * ) * Division ( / ) * Exponents (
double asterisks with no space in between: ** ... Ex: 2 ** 3 is 2 raised
to the power of 3.) * Modulus ( \% ...this finds the remainder after you
divide two numbers)

Here's a quick example:

\begin{itemize}
\tightlist
\item
  ten = 5 * 2
\end{itemize}

In the above example, "ten" is the \emph{name} of the variable. The
variable is then assigned the product of 5 and 2 (10).

Practice using the operations named above. Do the following: Declare a
variable named "first" and assign it two numbers that are added together
to equal 15.\\
Declare a second variable that is assigned the value 14. Print out both
variables using print().

Create a third variable called "mod" and assign it to be your first
variable modulo your second variable. Print out your third variable. It
should print out a 1.

    \begin{Verbatim}[commandchars=\\\{\}]
{\color{incolor}In [{\color{incolor}34}]:} \PY{c+c1}{\PYZsh{} Insert your code here}
\end{Verbatim}


    \subsubsection{Conditional Statements and
Booleans}\label{conditional-statements-and-booleans}

A \textbf{boolean} can only have the values "True" or "False", either
implied or explicitly stated. For example, study the following lines:

myBoolean = True mySecondBoolean = False

myBoolean == mySecondBoolean

In the first line of code above, I assigned myBoolean the value "True"
(Capitalization matters!). I then assigned mySecondBoolean the value
"False". If I printed these variables out, they'd print out "True" and
"False" respectively. This explicitly named either variable as either
True or False.

Now, in the third line, I used the double-equals (==) operator to check
if myBoolean equals mySecondBoolean. But True does NOT equal False,
therefore if I printed the third line out, it would read False. In the
third line of code, the True/False statement is implicit.

In the code cell provided below, declare your own variables and copy
what I've done above to see the code run for yourself. Then, try
changing the True/False values and see what difference it makes.

    \begin{Verbatim}[commandchars=\\\{\}]
{\color{incolor}In [{\color{incolor}35}]:} \PY{c+c1}{\PYZsh{} Insert code here.}
\end{Verbatim}


    Now let's look at conditional statements. In regular plain English,
conditional statements are written in an "if...then...else" format.
\emph{If} the condition is met, \emph{then} the second half of the
sentence is valid. \emph{Else} the other statement is valid. ("If it
snows heavily, then they may cancel school! Or else we'll have to attend
school this Monday.)

Conditional statements in programming work the same way. If the
condition is met, then the code below it will be executed. If it is not
met, then nothing happens and that piece of code is ignored. In Python,
the format for writing conditional statements looks like this:

if {[}statement here{]}: CODETOEXECUTE\_A

elif {[}statement{]}: CODETOEXECUTE\_B

else: CODETOEXECUTE\_C

**Be mindful of the indentations! They DO matter. Any code that is
indented will only be executed if the if statement above it is met.

If you only want one conditional statement, you can use just the first
block of code ( if ({[}statement here{]}): CODETOEXECUTE\_A ). If there
are two conditional statements, use the first and last block of code (
if ({[}statement here{]}): CODETOEXECUTE\_A and else: CODETOEXECUTE\_C
).

Look at the code provided below. Run it and look at the result. Which
conditional statement was met? Then, anipulate the code to make each
if-statement true in turn. (This means you'll hit "RUN" three times.
Once after every edit.)

    \begin{Verbatim}[commandchars=\\\{\}]
{\color{incolor}In [{\color{incolor}36}]:} \PY{n}{x} \PY{o}{=} \PY{l+m+mi}{4} 
         
         
         \PY{k}{if} \PY{n}{x} \PY{o}{==} \PY{l+m+mi}{2}\PY{p}{:} 
             \PY{n+nb}{print}\PY{p}{(}\PY{l+s+s2}{\PYZdq{}}\PY{l+s+s2}{x equals 2!}\PY{l+s+s2}{\PYZdq{}}\PY{p}{)}
         \PY{k}{elif} \PY{n}{x} \PY{o}{==} \PY{l+m+mi}{45}\PY{p}{:} 
             \PY{n+nb}{print}\PY{p}{(}\PY{l+s+s2}{\PYZdq{}}\PY{l+s+s2}{x equals 45!}\PY{l+s+s2}{\PYZdq{}}\PY{p}{)}
         \PY{k}{else}\PY{p}{:} 
             \PY{n+nb}{print}\PY{p}{(}\PY{l+s+s2}{\PYZdq{}}\PY{l+s+s2}{x equals 46}\PY{l+s+s2}{\PYZdq{}}\PY{p}{)}
\end{Verbatim}


    \begin{Verbatim}[commandchars=\\\{\}]
x equals 46

    \end{Verbatim}

    \subsubsection{Python Lists}\label{python-lists}

Inside Python, there's a data structure called a List. Lists hold any
number of any kind of variable. Here's how to declare a list in Python.

myFirstList = {[}{]}

Then, you can add variables to that list by using the append() function.
For example:

myFirstList.append(1) myFirstList.append(2) myFirstList.append(3)

This adds the numbers 1, 2, and 3 to your List. In order to go through
your lists, you can use the following format to print out individual
variables.

print(myFirstList{[}0{]}) print(myFirstList{[}1{]})
print(myFirstList{[}2{]})

\textbf{NOTE: Lists begin with 0 as the first index. So our "1" is
stored at index 0, "2" is stored at index 1, and so on.}

If you're trying to print the whole list, you can just print it like
this:

print(myFirstList)

This will print out all your variables that you've stored, in order.

Now you try: The code below is incorrect and incomplete. Add the
necessary pieces of code to make this code run correctly and print out
the list.

    \begin{Verbatim}[commandchars=\\\{\}]
{\color{incolor}In [{\color{incolor}37}]:} \PY{c+c1}{\PYZsh{}This code is incorrect. Fix the mistakes.}
         
         \PY{n}{myList} \PY{o}{=} \PY{p}{(}\PY{p}{)}
         
         \PY{n}{myList}\PY{o}{.}\PY{n}{append}\PY{p}{(}\PY{l+s+s2}{\PYZdq{}}\PY{l+s+s2}{String 1}\PY{l+s+s2}{\PYZdq{}}\PY{p}{)}
         \PY{n}{myList}\PY{o}{.}\PY{n}{append}\PY{p}{(}\PY{l+s+s2}{\PYZdq{}}\PY{l+s+s2}{String 2}\PY{l+s+s2}{\PYZdq{}}\PY{p}{)}
         
         \PY{n+nb}{print}\PY{p}{(}\PY{n}{myList}\PY{p}{[}\PY{l+m+mi}{2}\PY{p}{]}\PY{p}{)}
         
         
         
         \PY{c+c1}{\PYZsh{}The code should print out \PYZdq{}String 2\PYZdq{} followed by [\PYZsq{}String 1\PYZsq{}, \PYZsq{}String 2\PYZsq{}] on the next line. }
\end{Verbatim}


    \begin{Verbatim}[commandchars=\\\{\}]

        ---------------------------------------------------------------------------

        AttributeError                            Traceback (most recent call last)

        <ipython-input-37-dce20218bc0a> in <module>()
          3 myList = ()
          4 
    ----> 5 myList.append("String 1")
          6 myList.append("String 2")
          7 
    

        AttributeError: 'tuple' object has no attribute 'append'

    \end{Verbatim}

    \subsubsection{Dictionaries}\label{dictionaries}

A Dictionary is similar to a List, but works with two values (a key and
a value) instead of using an index. A Dictionary can be declared like
this:

favoriteColors = \{ "Jane" : "purple", "John" : "blue", "Mary" : "pink"

\begin{verbatim}
   }
   
\end{verbatim}

Copy/paste the code above and print out the result using
print(favoriteColors).

    \begin{Verbatim}[commandchars=\\\{\}]
{\color{incolor}In [{\color{incolor} }]:} \PY{c+c1}{\PYZsh{} This is the code they\PYZsq{}d have.}
        
        
        \PY{n}{favoriteColors} \PY{o}{=} \PY{p}{\PYZob{}}
               \PY{l+s+s2}{\PYZdq{}}\PY{l+s+s2}{Jane}\PY{l+s+s2}{\PYZdq{}} \PY{p}{:} \PY{l+s+s2}{\PYZdq{}}\PY{l+s+s2}{purple}\PY{l+s+s2}{\PYZdq{}}\PY{p}{,}
               \PY{l+s+s2}{\PYZdq{}}\PY{l+s+s2}{John}\PY{l+s+s2}{\PYZdq{}} \PY{p}{:} \PY{l+s+s2}{\PYZdq{}}\PY{l+s+s2}{blue}\PY{l+s+s2}{\PYZdq{}}\PY{p}{,}
               \PY{l+s+s2}{\PYZdq{}}\PY{l+s+s2}{Mary}\PY{l+s+s2}{\PYZdq{}} \PY{p}{:} \PY{l+s+s2}{\PYZdq{}}\PY{l+s+s2}{pink}\PY{l+s+s2}{\PYZdq{}}
               
               \PY{p}{\PYZcb{}}
        
        \PY{n+nb}{print}\PY{p}{(}\PY{n}{favoriteColors}\PY{p}{)}
\end{Verbatim}


    You can remove values from your Dictionary by using the .pop() method.
Example:

favoriteColors.pop("Jane")

This would remove Jane from your Dictionary. Try this in the cell above,
where you've pasted the Dictionary I gave you. Use the .pop() method
under the print() method. Then print the newly edited dictionary using a
second print() function. See what happens when you

NOw it's your turn! Create your own Dictionary and practice adding and
deleting key/value pairs. Print the result each time, to see how this
has changed your Dictionary.

    \begin{Verbatim}[commandchars=\\\{\}]
{\color{incolor}In [{\color{incolor} }]:} \PY{c+c1}{\PYZsh{} Insert your code here}
\end{Verbatim}


    That's all for this Section. Now you know all the basic concepts in
Python and are ready to head on over to Section 3!


    % Add a bibliography block to the postdoc
    
    
    
    \end{document}
